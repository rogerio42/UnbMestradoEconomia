\section{Tema}%
IMPACTO DO AUXÍLIO ALIMENTAÇÃO NO RENDIMENTO ACADÊMICO DOS DISCENTES: UM ESTUDO NA UNIVERSIDADE DE BRASÍLIA ENTRE 2014 E 2017.

\section{Contextualização}%

A Resolução nº 138/2012, da Reitoria da UnB, estabeleceu o Programa Bolsa Alimentação com o objetivo de garantir alimentação básica aos estudantes que se encontram em vulnerabilidade econômica com a finalidade de sua manutenção na UnB. Para ter direito ao benefício, o estudante deverá estar regulamente matriculado em disciplinas presenciais e ser identificado pela DDS como estudante do grupo de vulneráveis. O estudante deverá apresentar a identidade estudantil no acesso aos refeitórios do restaurante. Estes poderão se alimentar no RU e ter direito a três refeições diárias: desjejum, almoço e jantar. 
Observa-se que o PNAES na UnB ampliou o número de benefícios, beneficiados e implementou novos programas para a manutenção de estudantes economicamente vulneráveis. Neste sentido, 8,5\% dos estudantes regularmente matriculados na UnB recebem algum tipo de ajuda para se manterem na universidade, e a lei de cotas para a escola pública, implementada em 2012, reforça a necessidade do PNAES na UnB, pois, desde 2016, 50\% das vagas ofertadas serão para escolas públicas, o que acarretará o ingresso maior de estudantes com situação econômica vulnerável.
Conclui-se que o PNAES é fundamental para a manutenção dos vulneráveis economicamente nas IFES sendo um instrumento essencial para a mobilidade educacional com a finalidade de corrigir ou minimizar as discrepâncias socioeconômicas brasileiras por meio da educação. 
Apesar dessas políticas, os diversos critérios para diferentes programas e a falta de um acompanhamento em longo prazo impedem um diagnóstico de que tais medidas realmente são eficazes para o alcance dos objetivos de manutenção dos alunos na universidade e redução da evasão.

A necessidade de avaliar as políticas públicas, com destaque para as que são voltadas para a educação superior, sugere um anseio de conhecer os resultados das políticas de assistência social especificamente a de auxílio alimentação. A busca desses resultados deve ser contínua e sempre atualizada, deve também considerar a influência de características socioeconômicas e a eficácia do instrumento, com o objetivo de tornar as políticas mais eficientes.


A avaliação da questão proposta é tecnicamente desafiadora, tanto em termos de conhecimento requerido como de dados necessários para a análise. A despeito deste fato, existe a necessidade permanente de avaliar o desempenho das instituições. Embora seja de conhecimento comum a resistência das organizações em geral a qualquer forma de avaliação, talvez devido ao caráter punitivo sempre associado. Na última década uma soma de esforços governamentais, institucionais e da sociedade civil vem fomentando a avaliação de  políticas públicas, com destaque para as que são voltadas para a educação superior.


Surge, assim, um anseio de conhecer os resultados das políticas de assistência social. A busca desses resultados deve ser contínua e sempre atualizada, deve também considerar a influência de características socioeconômicas e a eficácia de seus instrumento de medição de avaliação de desempenho.

\section{Problema}%
A adesão ao plano de assistência estudantil no programa de auxílio alimentação adotado pela UnB entre 2014 e 2017 foi relevante para o rendimento acadêmico do discentes?

\section{Palavras Chaves}%
Politicas públicas, Universidades públicas, Programa de assistência estudantil, Restaurante universitário, Rendimento acadêmico, Auxílio Alimentação e Rendimento acadêmico.


\section{Objetivos}%

\subsection{Objetivo Geral}
Analisar no período de outubro de 2014 a outubro de 2017, se a adesão ou não ao plano de Assistência Estudantil adotado pela Universidade de Brasília foi relevante para o rendimento acadêmico dos discentes.


\subsection{Objetivo Específicos}
\begin{enumerate}
	\item Obter dados comparativas sobre a política de assistência estudantil alimentar;
	\item Avaliar o alcance da política de assistência estudantil alimentar implementada pela Universidade de Brasília;
	\item Comparar o desempenho acadêmico dos beneficiários e não beneficiários de assistência estudantil alimentar;
	\item Avaliar o rendimento acadêmico de discentes beneficiados pela assistência ao decorrer do tempo na Universidade de Brasília; 
	\item Analisar a evasão dos discentes beneficiados pela assistência estudantil alimentar em comparação com os demais;
	\item Sugerir medidas que possibilitem melhor desempenho na implementação da política de assistência estudantil da universidade, no que diz respeito ao alcance dos objetivos a que ela se propõe, se for o caso.
\end{enumerate}%

\section{Referencial teórico}

%%%%%%%%%%%%%%%%%%%%%%%%%%%%%%%%%%%%%%%%%%%%%%%%%%%%%%%%%%%%%%%%%%%%%%%%%%%%%%%%

\subsection{O ensino superior no Brasil}

De acordo com a constituição brasileira, a educação é direito fundamental, universal e inalienável, instrumento de formação da cidadania e meio de emancipação social. A educação, direito de todos e dever do Estado e da família, será promovida e incentivada com a colaboração da sociedade, visando o pleno desenvolvimento da pessoa, seu preparo para o exercício da cidadania e sua qualificação para o trabalho (CF/88, Art. 205).

Entretanto, apesar da intenção historicamente preconizada em todas as constituições brasileiras, a educação, por muito tempo, não foi um direito exercido, de fato, por todos os cidadãos brasileiros. As universidades públicas e gratuitas, até o final do último século, foram de uso privilegiado das elites do país.

De acordo com Alves (2002), o acesso a estas universidades públicas, em sua grande maioria federais, sempre foi privilégio das classes sociais mais altas, uma vez que os processos de seleção privilegiam aqueles alunos que, devido ao seu histórico de ensino fundamental e médio em escolas particulares de alta qualidade, estão mais preparados para os concursos vestibulares.

Com base nesses novos conceitos, instalou-se no Brasil o modelo europeu de educação superior definido na Declaração de Bolonha que, segundo Chauí (1999), sugere a adaptação dos currículos das universidades às demandas de empresas locais e às recomendações dos organismos multilaterais internacionais e de alguns preceitos do movimento gerencialista que tomou conta da administração pública atual, apesar dos resquícios burocráticos e patrimonialistas ainda tão presentes da administração pública brasileira.

Sugerindo que as estruturas dos governos estabeleçam um acordo com as instituições de ensino, visando à elevação dos níveis de acesso e permanência e à mudança do padrão de qualidade, no intuito de promover, consolidar, ampliar e aprofundar processos de transformação das universidades públicas federais para a expansão da oferta de vagas do ensino superior, promovendo a inclusão social e a formação adequada aos novos paradigmas social e econômico vigentes, conforme preconizam as políticas públicas de educação nacionais e as orientações multinacionais.

%%%%%%%%%%%%%%%%%%%%%%%%%%%%%%%%%%%%%%%%%%%%%%%%%%%%%%%%%%%%%%%%%%%%%%%%%%%%%%%%

\subsection{Políticas Públicas}

Segundo Farah (2011) e Trevisan e Bellen (2008), o planejamento ações de Políticas Públicas voltadas para educação superior no Brasil é recente, pois só ganharam notoriedade nos anos 1980, quando os desafios impostos pela redemocratização trouxeram à tona questões relativas ao poder e à política.










\section{Citações forçadas}%
 Obs: Com estou utilizando o LaTex para gerar esse documento, a única forma que encontrei neste momento de fazer as referências que não possuem vínculo com o texto foi forçando dessa maneira. Estou procurando aprender a trabalhar com o LaTex.


 Citações: 
   \cite{Bleicher2016PoliticasFederais},%
   \cite{MarreiroBarbosa2013DESEMPENHOHEMOGLOBINA},% 
   \cite{Santos2016ASocial},% 
   \cite{Machado2016DireitoUniversitarios},% 
   \cite{SilvaMachadoSistemaEducation},% 
   \cite{CarolinaLilideAssis2013ASBRASILEIRAS},% 
   \cite{VELOSOBARBOSA2015SOBREVIVERBENEFICIARIOS/AS},% 
   \cite{HARTMANNLEIBOVICH2015AvaliacaoUnB.},% 
   \cite{GuedesdaSilva2017EvasaoInstitucionais.},% 
   \cite{SERGIORIBEIROPINHO2017ANALISEFORTALEZA},% 
   \cite{FREITASDIASPINHEIRO2015AVALIACAOEDUCACIONAL},% 
   \cite{VIEIRADASILVA2016AVALIACAOUnB}% 
  

