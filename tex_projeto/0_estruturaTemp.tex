\section{Tema}%
IMPACTO DO AUXÍLIO ALIMENTAÇÃO NO RENDIMENTO ACADÊMICO DOS DISCENTES: UM ESTUDO NA UNIVERSIDADE DE BRASÍLIA ENTRE 2014 A 2017.
\cite{VIEIRADASILVA2016AVALIACAOUnB},% 

\section{Contextualização}%
A necessidade de avaliar as políticas públicas, com destaque para as que são voltadas para a educação superior. Surge um anseio de conhecer os resultados das políticas de assistência social especificamente a de auxílio alimentação. A busca desses resultados deve ser contínua e sempre atualizada, deve também considerar a influência de características socioeconômicas e a eficácia do instrumento, com o objetivo de tornar as políticas mais eficientes.


A avaliação da questão proposta é tecnicamente desafiadora, tanto em termos de conhecimento requerido como de dados necessários para a análise. A despeito deste fato, existe a necessidade permanente de avaliar o desempenho das instituições. Embora seja de conhecimento comum a resistência das organizações em geral a qualquer forma de avaliação, talvez devido ao caráter punitivo sempre associado, na última década uma soma de esforços governamentais, institucionais e da sociedade civil vem fomentando a avaliação de  políticas públicas, com destaque para as que são voltadas para a educação superior.


Surge, assim, um anseio de conhecer os resultados das políticas de assistência social. A busca desses resultados deve ser contínua e sempre atualizada, deve também considerar a influência de características socioeconômicas e a eficácia de seus instrumento de medição de avaliação de desempenho.

\section{Problema}%
Qual o impacto do auxílio alimentação no rendimento acadêmico dos discentes da universidade de Brasília entre 2014 a 2017?

\section{Palavras Chaves}%
Politicas públicas, Universidades públicas, Programa de assistência estudantil, Restaurante universitário, Rendimento acadêmico.


\section{Objetivos}%

\subsection{Objetivo Geral}
O objetivo geral do estudo é analisar, a partir dos dados obtidos do Sistema de Informações Acadêmicas de Graduação (SIGRA) e do Sistema de controle de acesso ao Restaurante Universitários (SISRU), no período de outubro de 2014 a outubro de 2017, se a adesão ou não ao plano de Assistência Estudantil adotada pela Universidade de Brasília foi relevante para o aumento do rendimento acadêmico dos discentes. Os resultados dessa pesquisa devem promover a discussão acerca do assunto e promover uma profunda reflexão sobre tais políticas públicas.

\subsection{Objetivo Específicos}
\begin{enumerate}
	\item Obter dados comparativas sobre a política de assistência estudantil alimentar;
	\item Avaliar a eficácia da política de assistência estudantil alimentar implementada pela Universidade de Brasília;
	\item Comparar beneficiários e não beneficiários de assistência estudantil alimentar;
	\item Avaliar o rendimento acadêmico de discentes beneficiados pela assistência ao decorrer do tempo na Universidade de Brasília; 
	\item Analisar rendimento acadêmico dos alunos no período de outubro de 2014 a outubro de 2017;
	\item Analisar a evasão dos discentes beneficiados pela assistência estudantil alimentar em comparação com os demais;
	\item Sugerir medidas que possibilitem melhor desempenho na implementação da política de assistência estudantil da universidade, no que diz respeito ao alcance dos objetivos a que ela se propõe, se for o caso.
\end{enumerate}%
\section{Citações}%
   Citações: 
   \cite{Bleicher2016PoliticasFederais},%
   \cite{MarreiroBarbosa2013DESEMPENHOHEMOGLOBINA},% 
   \cite{Santos2016ASocial},% 
   \cite{Machado2016DireitoUniversitarios},% 
   \cite{SilvaMachadoSistemaEducation},% 
   \cite{CarolinaLilideAssis2013ASBRASILEIRAS},% 
   \cite{VELOSOBARBOSA2015SOBREVIVERBENEFICIARIOS/AS},% 
   \cite{HARTMANNLEIBOVICH2015AvaliacaoUnB.},% 
   \cite{GuedesdaSilva2017EvasaoInstitucionais.},% 
   \cite{SERGIORIBEIROPINHO2017ANALISEFORTALEZA},% 
   \cite{FREITASDIASPINHEIRO2015AVALIACAOEDUCACIONAL},% 
   \cite{VIEIRADASILVA2016AVALIACAOUnB},% 
  

