\section{Tema}%
IMPACTO DO AUXÍLIO ALIMENTAÇÃO NO RENDIMENTO ACADÊMICO DOS DISCENTES: UM ESTUDO NA UNIVERSIDADE DE BRASÍLIA ENTRE 2014 E 2017.

\section{Contextualização}%

A Resolução nº 138/2012, da Reitoria da UnB, estabeleceu o Programa Bolsa Alimentação com o objetivo de garantir alimentação básica aos estudantes que se encontram em vulnerabilidade econômica com a finalidade de sua manutenção na UnB. Para ter direito ao benefício, o estudante deverá estar regulamente matriculado em disciplinas presenciais e ser identificado pela DDS como estudante do grupo de vulneráveis. O estudante deverá apresentar a identidade estudantil no acesso aos refeitórios do restaurante. Estes poderão se alimentar no RU e ter direito a três refeições diárias: desjejum, almoço e jantar. 
Observa-se que o PNAES na UnB ampliou o número de benefícios, beneficiados e implementou novos programas para a manutenção de estudantes economicamente vulneráveis. Neste sentido, 8,5\% dos estudantes regularmente matriculados na UnB recebem algum tipo de ajuda para se manterem na universidade, e a lei de cotas para a escola pública, implementada em 2012, reforça a necessidade do PNAES na UnB, pois, desde 2016, 50\% das vagas ofertadas serão para escolas públicas, o que acarretará o ingresso maior de estudantes com situação econômica vulnerável.
Conclui-se que o PNAES é fundamental para a manutenção dos vulneráveis economicamente nas IFES sendo um instrumento essencial para a mobilidade educacional com a finalidade de corrigir ou minimizar as discrepâncias socioeconômicas brasileiras por meio da educação. 
Apesar dessas políticas, os diversos critérios para diferentes programas e a falta de um acompanhamento em longo prazo impedem um diagnóstico de que tais medidas realmente são eficazes para o alcance dos objetivos de manutenção dos alunos na universidade e redução da evasão.

A necessidade de avaliar as políticas públicas, com destaque para as que são voltadas para a educação superior, sugere um anseio de conhecer os resultados das políticas de assistência social especificamente a de auxílio alimentação. A busca desses resultados deve ser contínua e sempre atualizada, deve também considerar a influência de características socioeconômicas e a eficácia do instrumento, com o objetivo de tornar as políticas mais eficientes.


A avaliação da questão proposta é tecnicamente desafiadora, tanto em termos de conhecimento requerido como de dados necessários para a análise. A despeito deste fato, existe a necessidade permanente de avaliar o desempenho das instituições. Embora seja de conhecimento comum a resistência das organizações em geral a qualquer forma de avaliação, talvez devido ao caráter punitivo sempre associado. Na última década uma soma de esforços governamentais, institucionais e da sociedade civil vem fomentando a avaliação de  políticas públicas, com destaque para as que são voltadas para a educação superior.


Surge, assim, um anseio de conhecer os resultados das políticas de assistência social. A busca desses resultados deve ser contínua e sempre atualizada, deve também considerar a influência de características socioeconômicas e a eficácia de seus instrumento de medição de avaliação de desempenho.

\section{Problema}%
A adesão ao plano de assistência estudantil no programa de auxílio alimentação adotado pela UnB entre 2014 e 2017 foi relevante para o rendimento acadêmico do discentes?

\section{Palavras Chaves}%
Politicas públicas, Universidades públicas, Programa de assistência estudantil, Restaurante universitário, Rendimento acadêmico, Auxílio Alimentação e Rendimento acadêmico.


\section{Objetivos}%

\subsection{Objetivo Geral}
Analisar no período de outubro de 2014 a outubro de 2017, se a adesão ou não ao plano de Assistência Estudantil adotado pela Universidade de Brasília foi relevante para o rendimento acadêmico dos discentes.


\subsection{Objetivo Específicos}
\begin{enumerate}
	\item Obter dados comparativas sobre a política de assistência estudantil alimentar;
	\item Avaliar o alcance da política de assistência estudantil alimentar implementada pela Universidade de Brasília;
	\item Comparar o desempenho acadêmico dos beneficiários e não beneficiários de assistência estudantil alimentar;
	\item Avaliar o rendimento acadêmico de discentes beneficiados pela assistência ao decorrer do tempo na Universidade de Brasília; 
	\item Analisar a evasão dos discentes beneficiados pela assistência estudantil alimentar em comparação com os demais;
	\item Sugerir medidas que possibilitem melhor desempenho na implementação da política de assistência estudantil da universidade, no que diz respeito ao alcance dos objetivos a que ela se propõe, se for o caso.
\end{enumerate}%

\section{Referencial Teórico}

Em razão da complexidade dos padrões de interação sociais envolvidos na formulação e na gestão das políticas públicas, os estudiosos dessas formas de ações coletivas organizadas têm procurado elaborar modelos e/ou referenciais analíticos capazes de capturar os elementos essenciais do processo de decisão que levaram a sua institucionalização. O problema é que no seu trabalho de hierarquização das variáveis relevantes, o analista está sempre sujeito ao risco de simplificar demais e perder grande parte dos aspectos essenciais dos determinantes e da dinâmica das políticas públicas. Para assinalar esses dilemas, uma breve apresentação de modelos desenvolvidos por diferentes áreas do pensamento econômico pode ser ilustrativa. 


%%%%%%%%%%%%%%%%%%%%%%%%%%%%%%%%%%%%%%%%%%%%%%%%%%%%%%%%%%%%%%%%%%%%%%%%%%%%%%%%

\subsection{Escolhas Coletivas no Quadro Paradigmático da Escolha Racional}

Começamos com a questão das escolhas coletivas no quadro paradigmático da escolha racional. Desenvolvido de forma elegante e sistemática por Arrows \cite{Arrow1970SocialValues}, essa abordagem repousa sobre um conjunto de hipótese bastante restritivo. Em primeiro lugar, supõe-se a existência de um agente central (Estado/Governo) perfeitamente racional e benevolente. Além disso, esse agente dispõe de todas as informações relevantes e tem o direito coletivo de implementar as políticas desejadas. O papel do governo, nesse modelo, é de maximizar o bem estar social tendo em vista o conjunto de preferências individuais. As políticas são, portanto, justificadas quando existe situação marcada por falhas de mercado. Porém, esse conjunto de hipótese que forma o núcleo duro do modelo da escolha racional gera uma série de problemas e questões analíticas. A mais conhecida foi desenvolvida pelo próprio Arrows e determina que não existe uma escolha social capaz de refletir perfeitamente as preferências individuais – trata-se do famoso teorema da impossibilidade. Além disso, o modelo pressupõe que o Estado age de forma benevolente, não levando em conta o fato de que a administração pública, por exemplo, pode agir de forma a maximizar sua utilidade em detrimento do interesse social. Existem também questões mais práticas. Por exemplo, como medir e internalizar as externalidades positivas e negativas quando há (ou quanto à)?? incerteza sobre os 4 custos incorridos? Como levar em conta demandas em situações onde não existem mercados para tais?


%%%%%%%%%%%%%%%%%%%%%%%%%%%%%%%%%%%%%%%%%%%%%%%%%%%%%%%%%%%%%%%%%%%%%%%%%%%%%%%%

\subsection{Escolha Pública mais Realista}

Para superar esses problemas teóricos um conjunto de autores vai procurar estabelecer um modelo de Escolha Pública mais realista proposto por Tullock \cite{Tullock1986TheBurocracy}. O ponto de partida dessa influente perspectiva está no reconhecimento da existência de diferentes indivíduos racionais e organizações com interesses divergentes. Os indivíduos, nessa corrente que afina os pressupostos neoclássicos, buscam maximizar suas funções de utilidade sujeitas a restrições. As organizações emergem da agregação de indivíduos com interesses comuns e, portanto, são voltadas para a proteção e promoção dos interesses individuais, embora os autores muitas vezes supõem que as organizações sejam capazes de desenvolver seus próprios interesses (representativos da convergência dos interesses individuais). As políticas públicas, nesse sentido, são o resultado de um processo político que busca alinhar as preferências dos agentes com os interesses das organizações e instituições. Por exemplo, os homens políticos estão motivados por (re)eleições e votarão políticas que favoreçam seus eleitores; a administração e a burocracia buscam influenciar o conteúdo das políticas para promover os objetivos de suas organizações; o alto escalão tentará influenciar políticas de interesses nacionais, etc.

%%%%%%%%%%%%%%%%%%%%%%%%%%%%%%%%%%%%%%%%%%%%%%%%%%%%%%%%%%%%%%%%%%%%%%%%%%%%%%%%

\subsection{Teoria da Escolha Pública as Políticas Públicas}

De modo geral, ainda que de forma simplificada, pode-se dizer que para a teoria da Escolha Pública as políticas públicas resultam da confrontação de interesses divergentes nos diversos mercados políticos que estruturam o sistema político como um todo. Porém, pouco se diz das regras institucionais que influenciam os padrões de interação desses mercados políticos. Ora, se as informações são assimétricas e os agentes potencialmente oportunistas, os mercados políticos operam com elevados custos de transações, isto é os custos vinculados da dificuldade de estabelecer padrões de cooperação entre os atores North\cite{North1990APolitics}. Na perspectiva neoinstitucionalista da escolha racional, a importância desses custos associados aos mercados políticos depende em grande parte dos arranjos institucionais, formais e informais, que estruturam os padrões de interação entre os diferentes participantes do jogo político. Nesse contexto, as instituições políticas têm um custo para a formulação de determinadas políticas públicas. Esses custos derivam: 

\begin{enumerate}%
	\item 1) do fato de que as instituições determinam quais são os atores relevantes, seus ganhos esperados, a arena onde interagem e a freqüência das interações e;
	\item 2) dos custos de transações políticos.
\end{enumerate}%


Segundo Alston \cite{Alston2006PoliticalBrazil}, no caso brasileiro, as políticas podem ser explicadas pelos padrões de interação entre o Presidente da República, os membros do Congresso e os demais atores capazes de interferir nesse jogo. Em função da pressão eleitoral, o Presidente apresenta uma relação de preferência hierárquica. No topo da agenda encontram-se as políticas que contribuam para fortalecer a estabilidade macroeconômica e o crescimento. Num nível inferior estariam políticas promovendo oportunidades econômicas e em seguida políticas visando a redução da pobreza. Os deputados e senadores, por outro lado, tendem a privilegiar políticas (setoriais, econômicas ou sociais) que trazem recursos para seus eleitores potenciais. Em função 6 das diversas preferências, os poderes Executivo e Legislativo procuram estabelecer relações que sejam benéficas a ambos. Assim, o foco do titular do governo está nas políticas macro (fiscal e monetária) e para alcançá-las pode utilizar políticas setoriais como moeda de troca no intuito de garantir votos no legislativo. 

%%%%%%%%%%%%%%%%%%%%%%%%%%%%%%%%%%%%%%%%%%%%%%%%%%%%%%%%%%%%%%%%%%%%%%%%%%%%%%%%

\subsection{A Importância das Instituições de Ensino Superior na Economia}

As Instituições de Ensino Superior são reconhecidas como mecanismos de desenvolvimento regional, não só pelo seu principal papel – educar os indivíduos – mas também devido à sua influência na região ou localidade, a qual compreende vários aspectos que não podem ser isolados, mas que estão fortemente relacionados (Smith, 2006)\cite{Smith2006TheSystem}. As Instituições de Ensino Superior são, portanto, instituições de elevada importância financeira e social nas regiões em que operam, garantindo oportunidades educacionais, económicas, sociais e culturais que de outra forma não existiriam na região. As IES não só criam oportunidades e empregos que ajudam a manter a região viva como podem trazer fundos para a região, através da sua capacidade em converter recursos em empreendimentos educacionais, de investigação, médicos e de serviços públicos. São, portanto, uma fonte poderosa de oportunidades (directas e indirectas) de emprego, de mão-de-obra altamente qualificada, de especialistas técnicos para os negócios locais e para atrair e reter investimentos \cite{Charney2003UniversityEconomy}, \cite{Goddard1987UniversitiesOverview}, \cite{Smith2006TheSystem} (Charney e Pavlakovich-Kochi, 2003; Carr e Roessner, 2002; Goddard, 1987; Smith, 2006).

%%%%%%%%%%%%%%%%%%%%%%%%%%%%%%%%%%%%%%%%%%%%%%%%%%%%%%%%%%%%%%%%%%%%%%%%%%%%%%%%







\section{Citações Forçadas}%
 Obs: Com estou utilizando o LaTex para gerar esse documento, a única forma que encontrei neste momento de fazer as referências que não possuem vínculo com o texto foi forçando dessa maneira. Estou procurando aprender a trabalhar com o LaTex.

 \subsection{Citações Pré-projeto}
    
   \cite{Bleicher2016PoliticasFederais},%
   \cite{MarreiroBarbosa2013DesempenhoHemoglobina},%
   \cite{Santos2016ASocial},%
   \cite{SilvaMachado2012SistemaEducation},%
   \cite{CarolinaLilideAssis2013ASBRASILEIRAS},%
   \cite{HartmannLeibovich2015AvaliacaoUnB},%
   \cite{GuedesdaSilva2017EvasaoInstitucionais},%
   \cite{Antonio2017AnaliseCeara},%
   \cite{Freitas2015AvaliacaoEducacional},%
   \cite{Vieira2016AvaliacaoBrasilia}%

