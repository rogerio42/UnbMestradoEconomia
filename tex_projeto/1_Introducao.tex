O presente projeto pretende analisar, a partir dos dados obtidos do \acrfull{SIGRA} e do \acrfull{SISRU}, no período de outubro de 2014 a outubro de 2017, se a adesão ou não ao plano de Assistência Estudantil adotada pela Universidade de Brasília foi relevante para o aumento do rendimento acadêmico dos discentes, fundamentando uma discussão acerca do cabimento ou não de tais políticas públicas.

Considerando que a Assistência Estudantil é compreendida em ações de assistência  nas áreas de moradia estudantil, alimentação, transporte, assistência à saúde, inclusão digital, cultura, esporte, creche, apoio pedagógico \cite{2007PORTARIA2007}, tem-se como foco da pesquisa a assistência em alimentação provida pelos restaurantes universitários distribuídos nos campi da Universidade de Brasília. Entende-se os restaurantes universitários como espaços geradores de atividade de ensino, pesquisa e extensão, dada sua função acadêmica, social e de convivência universitária, sendo um serviço essencial principalmente para os estudantes residentes ou de cursos de horário integral (FONAPRACE, 2017).

A questão que se coloca é: existiram diferenças no rendimento acadêmico entre discentes que receberam ou não assistência estudantil em forma de subsídios no auxílio alimentação na Universidade de Brasília entre outubro de 2014 a outubro de 2017? Tal política de assistência estudantil vêm logrando êxito para o seu objetivo primário?

%%%%%%%%%%%%%%%%%%%%%%%%%%%%%%%%%%%%%%%%%%%%%%%%%%%%%%%%%%%%%%%%%%%%%%%%%%%%%%%%

\section{Justificativa}%

A avaliação da questão proposta é tecnicamente desafiadora, tanto em termos de conhecimento requerido como de dados necessários para a análise. A despeito deste fato, existe a necessidade permanente de avaliar o desempenho das instituições. Embora seja de conhecimento comum a resistência das organizações em geral a qualquer forma de avaliação, talvez devido ao caráter punitivo sempre associado, na última década uma soma de esforços governamentais, institucionais e da sociedade civil vem fomentando a avaliação de  políticas públicas, com destaque para as que são voltadas para a educação superior.

Surge, assim, um anseio de conhecer os resultados das políticas de assistência social. A busca desses resultados deve ser contínua e sempre atualizada, deve também considerar a influência de características socioeconômicas e a eficácia de seus instrumento de medição de avaliação de desempenho.

%%%%%%%%%%%%%%%%%%%%%%%%%%%%%%%%%%%%%%%%%%%%%%%%%%%%%%%%%%%%%%%%%%%%%%%%%%%%%%%

\section{Objetivos}%

%%%%%%%%%%%%%%%%%%%%%%%%%%%%%%%%%%%%%%%%%%%%%%%%%%%%%%%%%%%%%%%%%%%%%%%%%%%%%%%%
\subsection{Objetivo Geral}
O objetivo geral do estudo é, assim, analisar, a partir dos dados obtidos do Sistema de Informações Acadêmicas de Graduação (SIGRA) e do Sistema de controle de acesso ao Restaurante Universitários (SISRU), no período de outubro de 2014 a outubro de 2017, se a adesão ou não ao plano de Assistência Estudantil adotada pela Universidade de Brasília foi relevante para o aumento do rendimento acadêmico dos discentes. Os resultados dessa pesquisa devem promover a discussão acerca do assunto e promover uma profunda reflexão sobre tais políticas públicas.

%%%%%%%%%%%%%%%%%%%%%%%%%%%%%%%%%%%%%%%%%%%%%%%%%%%%%%%%%%%%%%%%%%%%%%%%%%%%%%%%

\subsection{Objetivo específicos}

%%%%%%%%%%%%%%%%%%%%%%%%%%%%%%%%%%%%%%%%%%%%%%%%%%%%%%%%%%%%%%%%%%%%%%%%%%%%%%%%

\begin{enumerate}
	\item Obter dados comparativas sobre a política de assistência estudantil alimentar;
	
	\item Avaliar a eficácia da política de assistência estudantil alimentar implementada pela Universidade de Brasília;
	
	\item Comparar beneficiários e não beneficiários de assistência estudantil alimentar;
	
	\item Avaliar o rendimento acadêmico de discentes beneficiados pela assistência ao decorrer do tempo na Universidade de Brasília; 
	
	\item Analisar rendimento acadêmico dos alunos no período de outubro de 2014 a outubro de 2017;
	
	\item Analisar a evasão dos discentes beneficiados pela assistência estudantil alimentar em comparação com os demais;
	
	\item Sugerir medidas que possibilitem melhor desempenho na implementação da política de assistência estudantil da universidade, no que diz respeito ao alcance dos objetivos a que ela se propõe, se for o caso.
	
\end{enumerate}%

%%%%%%%%%%%%%%%%%%%%%%%%%%%%%%%%%%%%%%%%%%%%%%%%%%%%%%%%%%%%%%%%%%%%%%%%%%%%%%%%

\subsection{Revisão da Literatura}

%%%%%%%%%%%%%%%%%%%%%%%%%%%%%%%%%%%%%%%%%%%%%%%%%%%%%%%%%%%%%%%%%%%%%%%%%%%%%%%%

\subsection{O ensino superior no Brasil}

De acordo com a constituição brasileira, a educação é direito fundamental, universal e inalienável, instrumento de formação da cidadania e meio de emancipação social. A educação, direito de todos e dever do Estado e da família, será promovida e incentivada com a colaboração da sociedade, visando o pleno desenvolvimento da pessoa, seu preparo para o exercício da cidadania e sua qualificação para o trabalho (CF/88, Art. 205).

Entretanto, apesar da intenção historicamente preconizada em todas as constituições brasileiras, a educação, por muito tempo, não foi um direito exercido, de fato, por todos os cidadãos brasileiros. As universidades públicas e gratuitas, até o final do último século, foram de uso privilegiado das elites do país.

De acordo com Alves (2002), o acesso a estas universidades públicas, em sua grande maioria federais, sempre foi privilégio das classes sociais mais altas, uma vez que os processos de seleção privilegiam aqueles alunos que, devido ao seu histórico de ensino fundamental e médio em escolas particulares de alta qualidade, estão mais preparados para os concursos vestibulares.

Com base nesses novos conceitos, instalou-se no Brasil o modelo europeu de educação superior definido na Declaração de Bolonha que, segundo Chauí (1999), sugere a adaptação dos currículos das universidades às demandas de empresas locais e às recomendações dos organismos multilaterais internacionais e de alguns preceitos do movimento gerencialista que tomou conta da administração pública atual, apesar dos resquícios burocráticos e patrimonialistas ainda tão presentes da administração pública brasileira.

Sugerindo que as estruturas dos governos estabeleçam um acordo com as instituições de ensino, visando à elevação dos níveis de acesso e permanência e à mudança do padrão de qualidade, no intuito de promover, consolidar, ampliar e aprofundar processos de transformação das universidades públicas federais para a expansão da oferta de vagas do ensino superior, promovendo a inclusão social e a formação adequada aos novos paradigmas social e econômico vigentes, conforme preconizam as políticas públicas de educação nacionais e as orientações multinacionais.

%%%%%%%%%%%%%%%%%%%%%%%%%%%%%%%%%%%%%%%%%%%%%%%%%%%%%%%%%%%%%%%%%%%%%%%%%%%%%%%%

\subsection{Políticas Públicas}

Segundo Farah (2011) e Trevisan e Bellen (2008), o planejamento ações de Políticas Públicas voltadas para educação superior no Brasil é recente, pois só ganharam notoriedade nos anos 1980, quando os desafios impostos pela redemocratização trouxeram à tona questões relativas ao poder e à política.

%%%%%%%%%%%%%%%%%%%%%%%%%%%%%%%%%%%%%%%%%%%%%%%%%%%%%%%%%%%%%%%%%%%%%%%%%%%%%%%%

\subsection{Assistência Estudantil como Instrumento de Política Pública}

\emph{“A promulgação do Programa Nacional de Assistência Estudantil – PNAES, em 12 de dezembro de 2007, representa um marco histórico e de importância fundamental para a questão da assistência estudantil. Essa conquista foi fruto de esforços coletivos de dirigentes, docentes e discentes e representou a consolidação de uma luta histórica em torno da garantia da assistência estudantil enquanto um direito social voltado para igualdade de oportunidades aos estudantes do ensino superior público."}\footnote{VASCONCELOS, 2010, p. 608\url{http://www.unb.br}}

A \emph{assistência estudantil} possui orçamento específico, advinda diretamente dos orçamento das Universidades, elevando a assistência estudantil à categoria de política pública (Brasil, 2007). Isso sugere que a sociedade brasileira vem impondo a necessidade de maior eficiência e impacto nos investimentos do governo em políticas públicas, principalmente em programas sociais. A avaliação das políticas públicas se torna, assim, instrumento de fundamental importância para que se obtenham melhores resultados, não só no que diz respeito à melhor utilização e controle dos recursos aplicados, como também no que diz respeito à garantia do acesso a direitos sociais. A avaliação de determinada política pública deve ser motivada não só pela necessidade de garantia da correta aplicação dos recursos públicos a ela destinados, como também pela necessidade de verificação do cumprimento dos objetivos a que se propõe. “A avaliação de políticas públicas pode oferecer uma linha crítica de defesa contra tais deficiências, pela investigação sistemática da eficácia de políticas, programas e procedimentos” (XUN W., RAMESH M., HOWLETT M., 2014, p. 117).

“A promulgação do Programa Nacional de Assistência Estudantil – PNAES, em 12 de dezembro de 2007, representa um marco histórico e de importância fundamental para a questão da assistência estudantil. Essa conquista foi fruto de esforços coletivos de dirigentes, docentes e discentes e representou a consolidação de uma luta histórica em torno da garantia da assistência estudantil enquanto um direito social voltado para igualdade de oportunidades aos estudantes do ensino superior público."

%%%%%%%%%%%%%%%%%%%%%%%%%%%%%%%%%%%%%%%%%%%%%%%%%%%%%%%%%%%%%%%%%%%%%%%%%%%%%%%%

\subsection{Rendimento Acadêmico}

O indicador utilizado como medida de desempenho acadêmico é o Índice de Rendimento Acadêmico, IRA, utilizado na UnB como uma referência da progressão acadêmica do aluno. Esse índice tem caráter cumulativo e seu cálculo reflete as menções ponderadas pelos respectivos números de créditos das disciplinas cursadas, bem como a situação de permanência do aluno no fluxo do curso. A faixa de pontuação varia de zero a cinco (UNB, 2017).

%%%%%%%%%%%%%%%%%%%%%%%%%%%%%%%%%%%%%%%%%%%%%%%%%%%%%%%%%%%%%%%%%%%%%%%%%%%%%%%%

\subsection{Metodologia}

Entende-se que a pesquisa a ser realizada será quantitativa, descritiva, compreende que estes métodos serão os adequados para o processo de investigação em que se interessa descobrir e as relações existentes entre os aspectos e artefatos envolvidos.

%%%%%%%%%%%%%%%%%%%%%%%%%%%%%%%%%%%%%%%%%%%%%%%%%%%%%%%%%%%%%%%%%%%%%%%%%%%%%%%%

\subsection{Fontes de Dados}

As fontes de dados utilizadas para elaboração deste trabalho englobarão dois grupos principais:

Bancos de dados do Sistema de controle acesso ao Restaurante Universitários (SISRU), que possui entre seus requisito de implantação as regras de assistência estudantil;

Bancos de dados Sistema de Informações Acadêmicas de Graduação (SIGRA) que dispõe Índice de Rendimento Acadêmico (IRA), indicador utilizado como medida de desempenho acadêmico do discente. 

Ambos no seguinte intervalo de período: outubro de 2014 a outubro de 2017.

%%%%%%%%%%%%%%%%%%%%%%%%%%%%%%%%%%%%%%%%%%%%%%%%%%%%%%%%%%%%%%%%%%%%%%%%%%%%%%%%

\subsection{Análise dos Dados}

Após obtenção de acesso e validação dessas fontes de dados com a população de dados dos discentes contemplados e não contemplados com política assistenciais de auxílio alimentação, o objetivo será elaborar uma análise descritiva (médias, frequências e desvios-padrão) das amostras. Após a constatação da distribuição, o rendimento acadêmico dos grupos deverá ser comparado. 

Com o auxílio de métodos como a Análise por Envoltória de Dados (DEA, do inglês Data Envelopment Analysis) e ferramentas mineração de dados dotada de recursos de computação estatística para elaboração de modelo de análise de dados, inferência e simulações, será efetuado o cruzando dos dados, explorando ligações entre as fontes de informações, a fim de responder os questionamentos da pesquisa.

%%%%%%%%%%%%%%%%%%%%%%%%%%%%%%%%%%%%%%%%%%%%%%%%%%%%%%%%%%%%%%%%%%%%%%%%%%%%%%%%

\subsection{Plano de Trabalho}

Como o candidato a mestrando é servidor público da UnB, lotado no CPD, será dedicado o regime de tempo parcial. Dessa forma, uma vez que a referida instituição pública disponibiliza liberação de horário para os seus servidores de quadro efetivo cursarem disciplinas de cursos de pós-graduação, serão necessárias 12 (doze) horas semanais para a realização das disciplinas presenciais, mais 20 (vinte) horas de estudo, totalizando 32 (trinta e duas) horas semanais de dedicação. Após cursadas as disciplinas necessárias para completar os créditos exigidos para o título de Mestre, as horas dedicadas anteriormente serão convertida em dedicação exclusiva para conclusão do projeto de pesquisa.

%%%%%%%%%%%%%%%%%%%%%%%%%%%%%%%%%%%%%%%%%%%%%%%%%%%%%%%%%%%%%%%%%%%%%%%%%%%%%%%%





